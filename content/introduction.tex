\chapter{Introduction}

Boolean Satisfiability (SAT) solvers are widely used in formal verification, constraint solving, model checking, and other areas of computer science. Over decades of development, SAT solvers such as MiniSat have evolved into highly efficient tools and are often employed as the core reasoning engines within more complex systems, including Satisfiability Modulo Theories (SMT) solvers.

Modern applications increasingly require more than just solving isolated SAT instances. In many scenarios, the underlying problem evolves over time, making it necessary for SAT solvers to support incremental tasks efficiently by design. The IPASIR interface was introduced as a minimal and efficient API for incremental SAT solving, with which applications can add new constraints and assumptions before each solver call. However, when applications demand not only repeated calls to a SAT solver, but also tight integration and dynamic interaction throughout the solving process, IPASIR's design becomes limiting.

To address the limitation, the IPASIR-UP (User Propagators) interface was proposed. IPASIR-UP extends the IPASIR interface by enabling user-guided propagation, clause addition, and decision-making during solving. These capabilities are essential for achieving tight integration between SAT solvers and applications like SMT solvers. Despite its potential, IPASIR-UP has so far only been implemented on CaDiCaL. Its practicality and benefits need further demonstration.

This thesis presents MiniSatUP, a practical adaptation of the IPASIR-UP interface on MiniSat, which is a well-established and widely adopted SAT solver in both research and industry. The implementation focuses on minimizing modifications to the original codebase, preserving the existing control flow while enabling user propagation and clause addition during solving. This approach ensures that the solver remains maintainable, compatible, and efficient. As the second ever implementation of IPASIR-UP, this work demonstrates that IPASIR-UP can be incorporated into a mature CDCL SAT solver with minimal disruption.

In addition, we integrated MiniSatUP with cvc5, an open-source SMT solver, and evaluated the integration in terms of both correctness and performance. This integration illustrates the advantages of IPASIR-UP as a clean, incremental, and interactive interface, enabling effective SAT-SMT interaction without introducing measurable performance overhead, which serves as a concrete case study in applying IPASIR-UP to bridge SAT solvers and modern applications.

The thesis is structured in following chapters: \footnote{The author used large language models (LLMs) in a limited capacity to improve phrasing and clarity. All substantive content and original analysis remain the author's own.}

\begin{itemize}
  \item Background and Related Work
  \item Implementation
  \item Correctness and Performance Testing
  \item Summary and Future Work
\end{itemize}
