\chapter{Introduction}

Boolean Satisfiability (SAT) solvers are widely used in formal verification, constraint solving, model checking, and other areas of computer science. Over decades of development, SAT solvers such as MiniSat have evolved into highly efficient tools and are often employed as the core reasoning engines within more complex systems, including Satisfiability Modulo Theories (SMT) solvers.

Modern applications increasingly demand more than just solving standalone SAT instances. Applications such as SMT solvers require incremental solving and deeper interaction with the SAT solver. To support such use cases, the IPASIR interface was introduced as a minimal and efficient API for incremental SAT solving. However, IPASIR's design becomes limiting in scenarios where solvers must allow fine-grained control within the solving process, rather than merely between solver calls.

To address these limitations, the IPASIR-UP (User Propagation) extension was proposed. IPASIR-UP extends the IPASIR interface by enabling user-guided propagation, clause addition, and decision making during solving. These capabilities are essential for achieving tight integration between SAT solvers and SMT solvers. Despite its potential, IPASIR-UP has so far only been implemented in CaDiCaL, and used in its integration with cvc5. Its practicality and benefits need further demonstration.

This thesis presents a practical adaptation of the IPASIR-UP interface on MiniSat, a well-established and widely adopted SAT solver in both research and industry, resulting in MiniSatUP. The implementation focuses on minimizing modifications to the original codebase, preserving the existing control flow while enabling user propagation and clause addition during solving. This approach ensures that the solver remains maintainable, compatible, and efficient. As the second known implementation of IPASIR-UP, this work demonstrates that IPASIR-UP can be incorporated into a mature CDCL SAT solver with minimal disruption.

In addition, this work integrates MiniSatUP with cvc5, and the integration is evaluated for both correctness and performance. This integration illustrates the advantages of IPASIR-UP as a clean, incremental, and interactive interface, and shows measurable efficiency improvements in SAT-SMT interaction enabled by the adaptation, serving as a concrete case study in applying IPASIR-UP to bridge SAT solvers and mordern applications.

The thesis is divided into following chapters:

\begin{itemize}
  \item Background and related work
  \item Implementation
  \item Correctness and Performance Testing
  \item Summary and future work
\end{itemize}
