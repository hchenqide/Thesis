\chapter{Implementation}

This chapter provides a detailed account of the implementation of MiniSatUP, IPASIR-UP on MiniSat and the extension of cvc5 for intergrating with MinisatUP.

\section{Adaptation of MiniSat to IPASIR-UP}

% overview
MiniSat contains \code{core} solver and \code{simp} solver. In addition to the core CDCL solving loop performed by \code{core} solver, \code{simp} solver performs subsumption and variable elimation. In this work, only the \code{core} solver is modifed for adaption of IPASIR-UP.

Adaptation of MiniSat to IPASIR-UP involves several steps. First, the IPASIR-UP interface is added to header and the interaction with user propagater is inserted in the solving loop. Next, adding a clause provided by external propagater during solving is implemented. Then, adding a

% add ExternalPropagator to Minisat::Solver class
The \code{ExternalPropagator} declaration of the IPASIR-UP interface is added in the header file \code{Solver.h}, and the \code{external_propagator} member variable is added to \code{MiniSat::Solver}, which the solver could interact with during solving.

% put interaction code in CDCL loop
In each CDCL loop, after propagation and when there is no conflict, the interaction with user propagator is performed.

% assignment and backtrack notification
An index of current notified assignment on the trail is kept.

% external clause addition during solving

%% comparison with clause addition after solving
Adding a clause during solving is more challenging than adding a clause after each solving loop. A normal incremental clause addition, as is required in IPASIR interface, only happens when the solving is finished. At this time a variable is either assigned at root-level or unassigned. The clause to be added can be directly simplified by removing false literals. And after simplification of the clause, there are only 4 cases:

\begin{itemize}
  \item The clause is skipped when it contains a true literal.
  \item The result is set as UNSAT when the clause is empty.
  \item The unit is propagated when the clause contains only one literal.
  \item The clause is added to the clause set and watched by the first two literals.
\end{itemize}

But during solving, the literals in a clause can be assigned at different decision levels, and after addition of the clause, there might be multiple consequences that leads to different control flow of the current solving loop.

%% invariant of 2-watching scheme



%% interface description
Method \code{bool add_clause_solving(vec<Lit>& clause, bool forgettable, CRef& conflict, bool& propagate)} is added in core \code{MiniSat::Solver}, dedicated to support clause addition during solving. This method handles all scenarios that might happen when a new clause is added during solving. These scenarios are:

\begin{itemize}
  \item The clause is skipped.
  \item The clause is added, with the following possible consequences:
    \begin{itemize}
      \item No propagation or conflict analysis are needed.
      \item A unit propagation is needed.
      \item A backtrack and conflict analysis are needed.
    \end{itemize}
\end{itemize}

The input parameters \code {vec<Lit>& clause} and \code{bool forgettable} give the clause and tell whether it is forgettable. A forgettable clause is indicated by the external propagator when it provides a clause.

The return value \code {bool} indicates whether the clause is skipped (returning false) or added (returning true). If unit propagation is needed, the unit literal is assigned and added to trail, the output parameter \code{bool& propagate} is set true, and the program goes to the next loop for propagation. And when conflict analysis is needed, \code{CRef& conflict} ist set as the current added clause, and the program goes to \code{analyze}.

%% implementation

The input clause which is to be added is first

The new clause might be a tautology or root-satisfied clause, in this case it will be ignored, and \code{add_clause_solving} returns false. And when the clause is not ignored, the return value is true, and still there might be consequences brought by this clause. When it is a unit, it will cause unit propagation, which sets parameter \code{propagate} true. It might also be a conflict that will be reported through parameter \code{conflict}.

\todo{graph: original CDCL loop in minisat and changed loop as in project presentation}

% external propagation and lazy explaination

% check model

% decision

\section{Extension of cvc5 for integrating Minisat}

% duplicating the interface for cadical in cvc5

% adding cadical interface for minisatup

% linking minisatup in cvc5

\section{Development Environment and Tools}

% building and linking of MiniSat and cvc5

% debugging and testing
