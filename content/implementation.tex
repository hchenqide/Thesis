\chapter{Implementation}

This chapter provides a detailed account of the implementation of MiniSatUP (MiniSat with IPASIR-UP) and the extension of cvc5 for integrating with MiniSatUP.

\section{Adaptation of MiniSat to IPASIR-UP}

% overview and principle
Adaptation of MiniSat to IPASIR-UP involves several steps. First, the IPASIR-UP interface is added to the header file and interaction with user propagator is inserted in the solving loop. Then, adding a clause provided by user propagator during solving is implemented. Finally, adding an externally propagated literal from user and lazy explanation of the literal is implemented.

MiniSat contains \code{core} solver and \code{simp} solver. In addition to the core CDCL solving loop performed by \code{core} solver, \code{simp} solver performs subsumption and variable elimination. In this work, only the \code{core} solver is modified. And the modification is made as minimum as possible, without altering the major control flow of the original code, so that the solver's original behavior could be preserved while cleanly supporting the IPASIR-UP interface, enabling an easier inspection and comparison of the behavior change introduced by the modification. In total, around 400 lines of code are added including comments, and only around 10 lines of original code are changed.

% adding UserPropagator interface
Firstly, the \code{UserPropagator} class declaration of the IPASIR-UP interface is added in the header file \code{Solver.h}, and the \code{user_propagator} member variable is added to \code{MiniSat::Solver}, which can be set by \code{connect_user_propagator} for the solver to interact with during solving. If \code{user_propagator} is not initialized, then the \code{Solver} behaves just like the original solver.

% interaction within CDCL loop
%% assignment and backtracking notification
If \code{user_propagator} is initialized in \code{Solver}, it will interact with the solver in each CDCL loop after propagation and possible conflict analysis, before the next decision is made. At the beginning of each interaction, the user propagator is synchronized of the current assignments and possible backtracking. An index on the trail up to which the assignments are notified is maintained for this purpose. Then the user propagator is requested of possible external propagations and external clauses.

%% decision and check model
When there are no such propagations or clauses, or there's no consequent change of control flow after adding them, the user propagator is requested of the next decision literal, if there are still unassigned variables. The new decision is taken and notified to user propagator while going to the next loop of propagation. Or the user propagator is requested to do a full model check before returning SAT if every variable is assigned, so that a new solver call could be avoided if there should still be inconsistency of model on the user side. And when the model check fails, the loop is continued and user propagator is responsible to provide new propagations or clauses to be taken into account.

% external clause addition during solving
%% comparison with clause addition after solving
Secondly, adding a clause during solving is supported. Adding a clause during solving is more challenging than adding a clause after each solving loop. A normal incremental clause addition, as is required in IPASIR interface, only happens when the solving is finished. At this time a variable is either assigned at root-level or unassigned. The clause to be added can be directly simplified by removing false literals. And after simplification of the clause, there are only 4 cases:

\begin{itemize}
  \item The clause is skipped when it contains a true literal or is tautology.
  \item The result is set as UNSAT when the clause is empty.
  \item The unit is propagated when the clause contains only one literal.
  \item The clause is added to the clause set and watched by the first two literals.
\end{itemize}

But during solving, the literals in a clause can be assigned at different decision levels, and after addition of the clause, there might be multiple consequences that leads to different control flow changes in the current solving loop.

%% interface description
Method \code{bool add_clause_solving(vec<Lit>& clause, bool forgettable, CRef& conflict, bool& propagate)} is added in core \code{MiniSat::Solver}, dedicated to support clause addition during solving. This method handles all scenarios that might happen when a new clause is added during solving. These scenarios are:

\begin{itemize}
  \item The clause is skipped.
  \item The result is UNSAT.
  \item The clause is added, with the following possible consequences:
    \begin{itemize}
      \item No propagation or conflict analysis needed.
      \item Propagation needed.
      \item Conflict analysis needed.
    \end{itemize}
\end{itemize}

The input parameters \code {vec<Lit>& clause} and \code{bool forgettable} give the clause and tell whether it is forgettable. A forgettable clause is indicated by the user propagator when it provides a clause, and it can be removed during simplification of the clause set.

The returning \code{bool} indicates whether UNSAT is produced by this clause (returning true when UNSAT, otherwise false). If propagation is needed, the literal to propagate is assigned and added to trail, the output parameter \code{bool& propagate} is set true, and the program goes to the next loop for propagation. And when conflict analysis is needed, \code{CRef& conflict} is set as the current added clause, and the program goes to conflict analysis. When no propagation or conflict analysis is needed, none of the output parameters will be set, and the control flow goes further to check model or the next decision.

The interaction with user propagator and adding an external clause is inserted into the original CDCL loop, as figure~\ref{fig:flow} shows, where \code{interact} block handles notifying assignments, retrieving external clauses from user propagator and other callbacks, and \code{add} block adds a single external clause at one time which might cause different changes of control flow.

\begin{figure}[h!]
  \centering
  \includesvg[width=\textwidth]{flow.svg}
  \caption{Original CDCL loop in MiniSat and updated CDCL loop}
  \label{fig:flow}
\end{figure}

%% invariant of 2-watching scheme
In order to make sure the new clause is added without breaking the existing 2-watching scheme, an invariant across decision levels is established as following:

\begin{itemize}
  \item In decision level \code{l > 0}:
  \begin{itemize}
    \item A propagating clause is watched by the propagated literal (assigned true) and another literal (assigned false), both assigned on level l.
    \item A conflict clause is watched by two literals (both false) on level l.
  \end{itemize}
  \item After backtracking to a lower level \code{k < l}:
  \begin{itemize}
    \item The propagating clause or conflict clause at level l is still watched by the same two literals, now both unassigned.
  \end{itemize}
\end{itemize}

%% implementation
The clause to be added is first sorted by its literals and their current levels in the following order, so that when the first two literals become watching literals of the current clause, the clause will be still watched correctly by the two literals after backtracking according to the invariant.

\begin{center}
  true literals (level low - high) - unassigned literals - false literals (level high - low)
\end{center}

Similar to clause addition after solving, the clause is also simplified by removing root-level false literals. And when the clause contains a root-level true literal, or it is a tautology, it is skipped and the function returns without any additional consequence, and the next clause is requested until there's no more external clauses by the user propagator.

When the clause contains only one literal, it is assigned true after backtracking to root level, and propagated. When the clause contains at least two literals, it is added to the clause set, and is watched by the first two literals after sorting. Depending on the assignment status of the 2 watching literals, there can be 6 cases, and their levels when being assigned can introduce more sub-cases. Each case leads to its own outcome:

\begin{itemize}
  \item \code{false, false}: All literals in this clause are falsified so the clause a conflict. We check the levels of the watching literals (respectively a, b):
  \begin{itemize}
    \item \code{a = b}: This clause is a real conflict. We backtrack to level a/b and do conflict analysis.
    \item \code{a > b}: This clause just needs propagation.
  \end{itemize}
  \item \code{unassigned, false}: Propagation.
  \item \code{unassigned, unassigned}: No further action.
  \item \code{true, false}: When level \code{a > b}, backtrack and propagate, otherwise no further action.
  \item \code{true, unassigned}: No further action.
  \item \code{true, true}: No further action.
\end{itemize}

When the clause causes a propagation, we always backtrack to the level of the second literal, and then assign the first literal on this level, to make sure the levels of assignment for the two watching literals are always the same. Backtracking just to redo an existing assignment on a different level can be inefficient, since all the effort of propagation so far is wasted. It is also possible to just change the level of assignment for a variable while keeping the trail unchanged, which is known as out-of-order assignment or chronological backtracking, but this is tricky and for the current implementation, not yet necessary.

% external propagation and lazy explaination
In the last step of adapting MiniSat to IPASIR-UP, external propagation and lazy explanation of the externally propagated literal is supported. We inserted the callback function \code{cb_propagate} in the interaction code with user propagator to request propagated literals before requesting external clauses. The corresponding variable of the literal will be assigned accordingly with a tag \code{Clause_External} as a placeholder for its reason clause. If the literal is already assigned true, we skip it. And if it's already assigned false, which means that there is a hidden conflict, we get its reason clause immediately and add the clause reusing our previous function \code{add_clause_solving}.

During conflict analysis, reason clauses of assignments will be accessed for searching, and when a variable marked with \code{Clause_External} needs to be explained, we request the clause with callback \code{cb_add_reason_clause_lit} from user propagator. Like a clause being added during solving, a clause lazily added during conflict analysis is also sorted the same way and watched by the first two literals, though in this case it is either a root-level unit clause, or the first two watching literals are always true and false respectively, because the reason clause at this time should always be true with its only true literal being the literal to be explained.

Adding a true clause during conflict analysis can also cause propagation to be re-performed on a lower decision level, similar to the \code{unit} case and \code{true, false} case when adding an external clause. When the clause is a unit clause, a backtracking to root level is necessary. And when the watching literals of the clause are of different levels, it is also necessary to backtrack to the lower level and redo propagation to align their levels. The figure~\ref{fig:analyze} shows the updated CDCL loop for adding a reason clause lazily during conflict analysis.

\begin{figure}[h!]
  \centering
  \includesvg[width=0.6\textwidth]{analyze.svg}
  \caption{Updated CDCL loop with clause addition during conflict analysis}
  \label{fig:analyze}
\end{figure}

In order to achieve this without introducing many breaking changes for conflict analysis, we employ C++ exception to break from conflict analysis and raise the backtracking information expressed in a tuple of (destination level, propagating literal, reason clause reference). The analysis context, which stores the currently explored literals, will be cleared after catching the exception in the solving loop. With this trick, only 5 lines of code in \code{analyze} are changed, and 7 lines of code in the solving loop are added.

\section{Extension of cvc5 for integrating MiniSatUP}

% overview
After implementing MiniSatUP, we integrated it with cvc5 based on the existing cvc5 integration with CaDiCaL, where the \code{UserPropagator} is fully implemented to handle the interaction between cvc5 theory solvers and the SAT solver via IPASIR-UP. We copied the existing SAT solver class that implements IPASIR-UP for CaDiCaL, identified all functions it references the CaDiCaL solver, including those functions that are not part of IPASIR-UP but specific to CaDiCaL, and adapted MiniSatUP further to support these functions for a seamless integration.

The architecture of cvc5 is shown in figure~\ref{fig:cvc5}, which has an internal MiniSat tightly integrated as a CDCL SAT solver, and connects with CaDiCaL, which is referenced as an external package, via IPASIR-UP interface. MiniSatUP should be designed to reuse IPASIR-UP interface to connect to cvc5 alongside CaDiCaL.

\begin{figure}[h!]
  \centering
  \includesvg[width=0.5\textwidth]{cvc5.svg}
  \caption{Architecture of cvc5 with MiniSat and CaDiCaL through IPASIR-UP}
  \label{fig:cvc5}
\end{figure}

% copy cadical.h
In cvc5, we copied the existing \code{cadical.h} and \code{cadical.cpp}, which contain the CaDiCaL solver wrapper and the implementation of IPASIR-UP \code{UserPropagator}, and created respectively \code{minisatup.h} and \code{minisatup.cpp} for the integration. The symbols, as well as trace and statistics labels are renamed to MiniSatUP to distinguish from CaDiCaL. We didn't try to connect to and reuse the existing code directly because the IPASIR-UP implementation in cvc5 is still tightly bound with CaDiCaL, with a lot of symbols named after CaDiCaL and a lot of CaDiCaL-specific procedures. An abstraction of the common interface for both CaDiCaL and MiniSatUP as well as other possible SAT solvers that implement IPASIR-UP is still needed to be designed and developed.

% other interface functions
Now a direct compilation and linking is still not yet possible because of these CaDiCaL-specific functions which are not implemented by MiniSatUP. By searching in the source code for all referenced functions of CaDiCaL solver, a list of interfaces is created in \code{minisatup.h} and later implemented. They include:

\begin{itemize}
  \item IPASIR interfaces:
  \begin{itemize}
    \item \code{void add(int lit)}: Add a literal to the current clause, ending with 0. If the corresponding variable of the literal is not created yet, create the variable first.
    \item \code{void assume(int lit)}: Add a literal to the current assumption list.
    \item \code{int solve()}: Start solving. The assumptions will be cleared after solving. Return value 10, 20, 0 mean SAT, UNSAT and UNDEFINED respectively in accordance with CaDiCaL.
    \item \code{int val(int lit)}: Get the value of a literal from a SAT result. The return value is the literal if it's true, and the negation of the literal if it's false.
    \item \code{bool failed(int lit)}: Check if the literal as a part of the assumptions belongs to the final conflict clause for an UNSAT result.
  \end{itemize}
  \item IPASIR-UP interfaces:
  \begin{itemize}
    \item \code{void connect_user_propagator(UserPropagator *propagator)}: As explained before, the \code{UserPropagator} is added to and referenced by the solver. It is fully implemented by cvc5.
    \item \code{void add_observed_var(int var)}: Make the variable observed. We didn't implement this but only ensured the variable is constructed, as there's already a filtering of active variables in cvc5. This can also be simply implemented with a set containing all observed variables, or a bit field in the data for each variable indicating if it's observed.
    \item \code{void remove_observed_var(int var)}: Stop observing the variable. We didn't implement it either.
    \item \code{bool is_decision(int lit)}: Check if the literal is a decision literal. A decision literal has assignment level larger than 0 with no reason clause.
    \item \code{void phase(int lit)}: Set the phase of a variable. In MiniSat it is described as \code{polarity}.
  \end{itemize}
  \item CaDiCaL-specific interfaces:
  \begin{itemize}
    \item \code{int fixed(int var)}: Check if the variable has a fixed assignment. A variable with fixed assignment is assigned with level 0. In CaDiCaL, where out-of-order assignment is allowed, the variable could be assigned with actual level 0 on a non-zero decision level. But in MiniSatUP, which doesn't support out-of-order assignment, it is always assigned on decision level 0.
    \item \code{void terminate()}: Terminate the solving process asynchronously. We set a flag in solver and the solver checks it in every loop and exit if it is set.
    \item \code{void connect_terminator(Terminator *terminator)}: Connect the \code{Terminator} to sat solver. The \code{Terminate} class contains only one callback function \code{virtual bool terminate()} that will be called synchronously by the solver to check if it needs to terminate. This provides another way of terminating which works synchronously in addition to \code{void terminate()}.
    \item \code{void connect_learner(Learner *learner)}: Connect a clause learner to the solver. A callback function \code{virtual void learn(int lit)} from the \code{Learner} class is for the solver to notify the user application about a learned clause generated from conflict analysis. The learned clause can be used to guide user applications, here in cvc5 the theory solvers, for further decision-makings.
    \item \code{void connect_listener(FixedAssignmentListener *listener)}: Add a fixed-assignment listener. \code{FixedAssignmentListener} has one callback function \code{virtual void notify_fixed_assignment(int)}, where literals with fixed assignments are notified eagerly whenever there's an assignment, in comparison to \code{notify_assignment} in IPASIR-UP interface. This is a synchronous version for the function \code{int fixed(int var)}. In MiniSatUP this is not implemented either.
  \end{itemize}
\end{itemize}

We added declaration of these interface functions first without implementing them, just to make sure they can be linked correctly with cvc5. Then the functions are implemented only when errors occur, and are tested immediately after each implementation, so that only the necessary functions are included, and the behavioral change that each function introduces can be analyzed and understood. In the end, some functions didn't get implemented since they are never called or referenced, including some IPASIR-UP functions.

% linking minisatup in cvc5
cvc5 is built with CMake and its dependencies are added through \code{find_package} command in CMake. We wrapped MiniSatUP in a CMake package for cvc5 to be able to reference and link with it. And option \code{--sat-solver=minisatup} is also supported, so that MiniSatUP can be selected by cvc5 along with its internal MiniSat and CaDiCaL for parallel experimenting and testing.
