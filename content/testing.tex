\chapter{Correctness and Performance Testing}

This chapter presents the correctness testing conducted on MiniSatUP and its integration with cvc5. Additionally, performance testing carried out on cvc5 after the intergration demonstrates performance improvements on several theories.

\section{Correctness}

% overview
In developing a software that does formal proofs and verification, its own correctness is vitally important. Testing techniques like fuzzing and regression tests are employed in the development of many. In this work, a fuzzer for the IPASIR-UP interface is developed during implementation of MiniSatUP, to ensure the functionality of the interface is fully achieved by the implementation, as well as its correctness.

% minisat fuzzer
%% connection
The fuzzer takes a CNF file as input, which can be generated by CNFuzz for the randomness, and splits the clauses into two parts, with the first part given to the solver initially, and the rest added via IPASIR-UP interface. A \code{UserPropagator} implementing IPASIR-UP interfaces will be set to the solver for it to interact with.

%% external clause
An external clause is provided with a certain probability on each interaction. And a statistic on each case of adding a new clause, is  this ensures all cases when adding an external clauses is hit and tested.

%% external propagation and lazy propagation
The lazy propagation interface is also implemented by checking if the next clause to be added is actually a unit, and if so, the unit literal is returned while the clause is saved in a map to be retrieved in lazy explaination.

% bugs before cvc5 integration
The fuzzer has helped to identify some bugs in the early stages of implementing MiniSatUP, like an error in the predicate for sorting a clause, and another issue related with the timing of attaching user propagator to the solver when some assignments are not informed.

The implementation is connected and tested with a fuzzer before integrating in cvc5.

This fuzzer . It doesn't cover the interfaces like decisions and assumptions. And such functionalities are not essential in . After integrated with cvc5, these interfaces are fully tested and a few more bugs appeared.

% cvc5 make check and bugs after cvc5 integration

After integration with cvc5, several more bugs are found in MiniSatUP which were not discovered with the fuzzer alone. A bug in the fuzzer that calls \code{clauses.front()} without first checking if \code{clauses} is empty is only discovered after disabling the compiler flag \code{-fsanitize=address} that doesn't work with cvc5.

As a , cvc5 has its own test set that is evolving along with the development.

% bug classificaiton by severity, difficulty to find/fix

In summary, these bugs 

\section{Performance}

% benchmarks

We performed performance test on cvc5 with the dataset.

% result and comparison with cadical/minisat

no discrepancy occurred
