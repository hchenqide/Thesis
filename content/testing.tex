\chapter{Correctness and Performance Testing}

This chapter explains correctness testing performed on both MiniSatUP and its integration with cvc5. And a performance testing is performed on cvc5 with MiniSatUP in comparison with original cvc5 and cvc5 with CaDiCaL.

\section{Correctness}

% minisat fuzzer

A fuzzer for the IPASIR-UP interface is developed during implementation of MiniSatUP, to ensure the functionality of the interface is fully achieved by the implementation, as well as its correctness. The fuzzer takes a CNF file as input, which is generated by CNFuzz, and splits the clauses into two parts, with the first part given to the solver initially, and the rest added via IPASIR-UP interface. A fully functional \code{UserPropagator} implementing IPASIR-UP interfaces is interacting with the solver, providing a clause with a certain probability on each interaction. This ensures all cases when adding an external clauses is hit and tested. The lazy propagation interface is also implemented by checking if the next clause to be added is actually a unit, and if so, the unit literal is returned while the clause is saved in a map to be retrieved in lazy explaination.

% bugs before cvc5 integration

The fuzzer has helped to identify some bugs in the early stages of implementing MiniSatUP, like an error in the predicate for sorting a clause, and another issue related with the timing of attaching user propagator to the solver when some assignments are not informed.

% cvc5 make check and bugs after cvc5 integration

After integration with cvc5, several more bugs are found in MiniSatUP which were not discovered with the fuzzer alone. A bug in the fuzzer that calls \code{clauses.front()} without first checking if \code{clauses} is empty is only discovered after disabling the compiler flag \code{-fsanitize=address} that doesn't work with cvc5.

% bug classificaiton by severity, difficulty to find/fix

\section{Performance}

% benchmarks

no discrepancy occurred

% result and comparison with cadical/minisat

